\documentclass[12pt]{article}
 
\usepackage[T1]{fontenc}     
\usepackage[utf8]{inputenc}  
\usepackage[magyar]{babel}   
\usepackage{lmodern}
\usepackage{caption}        
 
\usepackage{amsmath,amssymb} 
\usepackage{amsthm}          
\usepackage{graphicx}  
\usepackage{hyperref}  
\usepackage[export]{adjustbox}
\usepackage{wrapfig}
\usepackage{float}
\usepackage[a4paper, left=35mm, right=25mm, top=25mm, bottom=25mm]{geometry}

\DeclareMathOperator*{\argmax}{argmax}    

\newtheorem{definition}{Definíció}[section]

\title{Covid-19-cel kapcsolatos tévinformációk azonosítása gépi tanulással}
\author{Köller Donát, BME TTK BSc. Szakdolgozat}
\date{\today}           
\captionsetup[table]{position=bottom}            
 
\linespread{1.25}

\begin{document}

\tableofcontents


\section{Bevezetés}

\section{A kognitív tesztről}

\section{Adatprepació}

\subsection{Az adatok beszerzése, adattáblák jellemzése}



\subsection{Adattisztítás}

A felderítő és modellezési fázisban használt adattáblát a rendelkezésre álló táblák megfelelő mértékű tisztításából és összeillesztéséből nyertük.

A kognitív eredményeket tartalmazó adattábla részletesen tartalmazott információt egyrészt minden hallgatóról (hova valósi, emelt érettségit tett-e matematikából, reál tagozatos volt-e, milyen szakra és tankörbe jár), másrészt a hallgató teszteredményéről is (mennyi idő alatt töltötte ki a tesztet, mely kérdéseket válaszolta meg jól, milyen lett a nyelvi és matekos teljesítménye), illetve tartalmazott néhány, a teszthez kapcsolódó egyéb információt is (például a teszthez használt edubase jelszó, felhasználónév). Természetesen nekünk ennyi adat nem kell, úgyhogy ebből az adattáblából jó pár irreleváns oszlopot ki kellett szűrnünk. Amelyeket meghagytunk, azok az alábbiak: a hallgató neve (ez már elég volt, hogy csak ez alapján fűzzük össze a táblákat) és Neptun kódja; emelt érettségit tett-e matematikából; reál/matematika tagozatos volt-e; szak és tankör; az elért pont és százalékos teljesítmény a nyelvi és matekos részben, valamint összességében. Problémát jelentett még, hogy a ’Szak’ mezőben mindenki másképp írta be azt, hogy melyik szakon tanul, így ezt szabványosítani kellett, ha később szakok szerint akartunk vizsgálódni. Erre a feladatra külön Python kódot írtunk, és ha volt olyan mező, ahol nem tudtuk eldönteni, hogy mi lenne az oda tartozó érték (például mert ’VBK’ volt odaírva), arra bevezettünk egy globális ’UNKNOWN’ változóértéket, azonban szerencsére ilyenből kevés volt. Kicsit még tisztítani kellett a ’Tankör’ értékeken is, de mivel ilyenből kevés volt, ezt manuálisan is meg tudtuk tenni.
A 0. ZH eredményeket tartalmazó tábla szerencsére ennél jóval kisebb volt, csak a hallgató nevét, Neptun kódját, képzés nevét, illetve kódját, felvétel évét, valamint a ZH eredményt tartalmazta. Ebből értelemszerűen csak a névre és az eredményre volt szükségünk, a többi elhagyható.

A felvételi pontszámokat bontva (hozott pont, érettségi, többletpont) tartalmazó tábla hallgatók nevén és születési dátumán kívül tartalmazott még pár, a felvételi eljáráshoz és felvételi döntéshez kapcsolódó adatot, illetve a ponthatárt. Ebből a táblából csak a név és a pontokat tartalmazó oszlopok kellettek, a többit elvethettük.

A Matematika A1a jegyeket tartalmazó táblával már több dolgunk volt, mint az előző kettő esetben. Először is minden személyhez több rekord tartozhatott, legalább egy az A1a jegyhez, lehetett még egy az A2c jegyhez (nyilván csak azoknak, akik elvégezték az A1a-t, és ott maradtak az egyetemen), illetve, ha egy korábbi, nem a végleges jegyet eredményező vizsgán egy hallgató megbukott, akkor ahhoz is tartozott egy rekord. Az attribútumok között a hallgató nevén, Neptun kódján és az osztályzatán kívül szerepelt még a felvétel éve, képzés neve, kódja, státusz ID (Aktív, elbocsátott stb.), Pénzügyi státusz (állami/önköltséges), a tantárgy neve, kódja, kreditértéke, jegy típusa, bejegyzés dátuma, illetve, hogy elismert és hogy érvényes-e az adott jegy. Ezekből az adatokból nekünk csak a hallgató nevére és jegyértékeire volt szükségünk, ráadásul olyan formában, hogy minden sor egy hallgatóhoz tartozzon, és az oszlopok a tantárgyakból szerzett jegyeket tartalmazzák. Ehhez először Pythonban kiszűrtük az irreleváns oszlopokat, majd ’crosstab’-eléssel a kívánt formára hoztuk az adatokat, ahol még ügyelni kellett arra, hogy a korábbi vizsgajegyek ne kerüljenek bele, tehát minden hallgatóhoz tárgyanként csak egy jegy tartozzon. Ezen kívül még, mivel az érdemjegyek szövegesen voltak megadva’, azokat számszerűvé alakítottuk, hogy majd a későbbiekben könnyebb legyen velük dolgozni.

Egy külön adattábla tartalmazta még a kognitív eredményeknél a matekos eredményt blokkokra lebontva, amely valójában az elsőként tekintett adattáblának volt egy egyszerűsített, kevesebb attribútummal bíró változata. Ebből az adattáblából csak a hallgatók nevére, illetve a blokkonkénti teljesítményre volt szükségünk, a többit elhagytuk.

Így már rendelkezésünkre állt az összes tábla, egyenként tisztítva, és már csak az összefűzés volt hátra, amit R-ben könnyen meg tudtunk tenni, valamint még a végén rendeztük az oszlopok sorrendjét, hogy az adathalmaz logikus szerkezetű legyen. Fontos megjegyezni azonban, hogy nem minden elsőéves írt abban az évben kognitív tesztet, így összefűzés során (ahol valójában ’inner-join’-oltunk) kevesebb sorunk lett, mint ahányan abban az évben a BME VBK karára felvételt nyertek.
A legvégén az így kapott adathalmaz 231 rekorddal és 21 oszloppal rendelkezett, amelyek között még esetlegesen szűrtünk különböző algoritmusok használata során.


\section{Felderítő adatelemzés a két évben}


\section{Modellépítés az elsőéves teljesítményekre}

\subsection{Osztályozó algoritmusok}

\subsubsection{kNN}

\subsubsection{Lineáris regresszió}

\subsubsection{Naive Bayes}

\subsubsection{Gradient Boosting}

\subsubsection{Logisztikus regresszió}

\subsubsection{SVM}


\subsection{Modellek és metodológia}

A modellezési fázis megkezdése előtt a folytonos változók kvantilis alapú 0-1 skálázásnak lettek alávetve, hogy a távolság alapú osztályozók jobban teljesítsenek. Összesen kétféle eredmény alaposabb vizsgálatára összpontosítottunk mindkét évben: a "Matematika A1a - Analízis" tárgyból kapott érdemjegyre illetve az első félév végén megállapított kumulált tanulmányi átlagra. A kettőből az előbbi különös fontossággal bír, ugyanis ebből a tantárgyból igen magas a lemorzsolódók aránya minden karon. A matematika érdemjegy prediktálására (pontosabban érdemjegycsoport prediktálására) kétféle modell került felvázolásra: egy 3 csoport modell, illetve egy 2 csoport modell.

\subsection{A 3 csoport modell}

Az 3 csoport modellen belül a jegyek az alábbi 3 csoportba lettek besorolva: 2-es címkéjű csoportba kerültek a Jeles illetve Jó érdemjegyek, 1-es címkéjűbe a Közepes és Elégséges érdemjegyek, míg az Elégtelen érdemjegyek alkotják a 0-ás címkéjű csoportot. A 3 csoportot egyszerűen úgy is megfogalmazhatjuk mint a jól teljesítő, a rosszul teljesítő teljesítő és a nem megfelelt hallgatók csoportja. A prediktálás során ezen 3 kategóriába szeretnénk minél pontosabban besorolni a rekordokat.

A modellen belül külön vizsgáltuk a hallgató eredményeket szakokra bontva és együttvéve egyaránt. A szakokra bontás esetén csupán a Vegyészmérnök illetve Biomérnöki szakok eredményeit néztük, ugyanis a Környezetmérnök hallgatókról nem állt rendelkezésre elég adat. 

\subsection{A 2 csoport modell}

A 2 csoport modellben az érdemjegyek felosztása az alábbi: 1-es címkéjű csoportba kerültek a Jeles, Jó, Közepes érdemjegyek, míg a 0-ás csoportba az Elégséges és Elégtelen jegyek. A 3 csoport modellhez hasonlóan itt is további két almodellt vizsgáltunk aszerint, hogy a jegyeket szakokra bontva prediktáljuk-e vagy sem. 

\subsection{Főkomponens analízis (PCA)}

Az egyes modellek és almodellek esetén főkomponens analízist is alkalmaztunk, amelynek az a lényege, hogy a rendelkezésre álló változókból kevesebb, új változókat hozunk létre miközben arra törekszünk, hogy a folyamat során elveszített információmennyiség minimális legyen.

\subsection{Kumulált átlag prediktálása és modellezése}

A kumulált átlag prediktálásánál csak lineáris regressziót használtuk és annak a vizsgálatára összpontosítottunk, hogy az egyes években a vázolt regressziós modelleknél melyek a legfontosabb változók, és hogy ezek milyen mértékben és irányban változtatják a predikciót.

\subsection{Optimalizálás}

\section{A modellezés eredményei}




\begin{thebibliography}{9}
\end{thebibliography}

\end{document}

