\documentclass[12pt]{article}
 
\usepackage[T1]{fontenc}     
\usepackage[utf8]{inputenc}  
\usepackage[magyar]{babel}   
\usepackage{lmodern}
\usepackage{caption}        
 
\usepackage{amsmath,amssymb} 
\usepackage{amsthm}          
\usepackage{graphicx}  
\usepackage{hyperref}  
\usepackage[export]{adjustbox}
\usepackage{wrapfig}
\usepackage{float}
\usepackage[a4paper, left=20mm, right=20mm, top=15mm, bottom=25mm]{geometry}

\usepackage{gillius}

% Set spacing
\usepackage{setspace}



% Add clickable links


\onehalfspacing


\linespread{1.3}

\title{Elsőéves hallgatók pandémia előtti és alatti bemeneti adatainak elemzése modern adattudományi eszközökkel}

\begin{document}

\maketitle



Az oktatás fejlesztése és korszerűsítése mindig magas prioritású feladat volt. A néhány éve kitört koronavírus járvány miatt az oktatás a világ nagy részében távolléti formára állt át. Ebben az környezetben a diákok és hallgatók teljesítménye jól tapasztalható változást mutatott. Ennek fényében tekintettük célszerűnek a Budapesti Műszaki és Gazdaságtudományi Egyetem Vegyészmérnöki és Biomérnöki Karának 2019-es és 2021-es elsőéves hallgatóinak bemeneti adatainak és eredményeinek vizsgálatát adattudományi módszerekkel.

Kutatásunk célja adott bemeneti mutatók és a hallgatók további sikeressége közötti viszonyok feltárása és az utóbbi prediktálása a bemeneti adatok alapján, valamint az így kapott eredmények összevetése a két évben. Egy jól előrejelző modell birtokában a hallgató későbbi teljesítménye a tanulmányai kezdetén megjósolható lenne, így a lemorzsolódásban veszélyeztetett hallgatók számára időben megfelelő segítséget lehetne  nyújtani.
% szórásdiagramok
Adattudományi eszközök széles spektrumát használtuk a kutatás valamennyi szakaszában. Az adatvizualizációban kiemelhetők a Sankey-féle folyamatábrák, melyeken látványosak az évek közti különbségek. Az adatok klaszterezésénél háromféle módszer lett kipróbálva és összevetve. A legjobb predikciós pontosság érdekében szintén több modellt építettünk fel és hasonlítottunk össze.

Már az adatvizualizáció során sikerült az adatok között némi összefüggéseket és az évek között különbségeket feltárni. Ennek ellenére az adatok nem voltak jól klaszterezhetők. Sikeres lett a prediktív modellezés, a korlátozott adatmennyiség ellenére magas pontossággal jelezték előre modelljeink az osztályokat.



\end{document}