\documentclass[12pt]{article}
 
\usepackage[T1]{fontenc}     
\usepackage[utf8]{inputenc}  
\usepackage[magyar]{babel}   
\usepackage{lmodern}
\usepackage{caption}        
 
\usepackage{amsmath,amssymb} 
\usepackage{amsthm}          
\usepackage{graphicx}  
\usepackage{hyperref}  
\usepackage[export]{adjustbox}
\usepackage{wrapfig}
\usepackage{float}
\usepackage[a4paper, left=20mm, right=20mm, top=15mm, bottom=25mm]{geometry}

\usepackage{gillius}

% Set spacing
\usepackage{setspace}



% Add clickable links


\onehalfspacing


\linespread{1.3}

\title{Elsőéves hallgatók pandémia előtti és alatti bemeneti adatainak elemzése modern adattudományi eszközökkel}

\begin{document}

\maketitle



Az oktatás fejlesztése és korszerűsítése mindig magas prioritású feladat volt. A néhány éve kitört koronavírus járvány miatt az oktatás a világ nagy részében távolléti formára állt át. Ebben az környezetben a diákok és hallgatók teljesítménye jól tapasztalható változást mutatott. Ennek fényében tekintettük célszerűnek a Budapesti Műszaki és Gazdaságtudományi Egyetem Vegyészmérnöki és Biomérnöki Karának 2019-es és 2021-es elsőéves hallgatóinak bemeneti adatainak és eredményeinek vizsgálatát adattudományi módszerekkel.

Kutatásunk célja a hallgatók adatai közötti viszonyok feltárása és a hallgatók további eredményeinek előrejelzése ezek alapján, illetve mindezek összehasonlítása a két évben. Egy jól előrejelző modell birtokában a hallgató későbbi teljesítménye a tanulmányai kezdetén megjósolható lenne, így a lemorzsolódásban veszélyeztetett hallgatók számára időben megfelelő segítséget lehetne  nyújtani.

Adattudományi eszközök széles spektrumát használtuk a kutatás valamennyi szakaszában. Az adatvizualizációban az oszlop- és szórásdiagramok mellett kiemelhetők a Sankey-féle folyamatábrák, melyeken látványosak az évek közti különbségek. Kísérletet tettünk a hallgatók csoportosítására az adatok alapján, amelyhez háromféle klaszterezési módszert alkalmaztunk. A hallgatók sikerességének előrejelzéséhez szintén számos gépi tanulási algoritmust használtunk. Az első féléves matematika jegyekből képzett jegycsoportokra különböző osztályozó, a félév végi átlagra pedig többféle regressziós modellt építettünk és hasonlítottunk össze.

Az adatvizualizáció során sikerült összefüggéseket találni az adatokban, többek közt kirajzolódott az emelt szintű érettségi pozitív hatása a további eredményekre, illetve látványos különbségeket mutattunk ki az eredmények eloszlásában a két évben. A klaszterezés nem hozott kielégítő eredményt. A prediktív modellezés során a legjobb modellek 80\% fölötti hatásfokot értek el az érdemjegyekre nézve, a regresszió esetén pedig a legjobb átlagos négyzetes eltérés 0.4 körül mozgott mindkét évben.

Összességében kijelenthető, hogy a hallgatók eredményessége romlott, de ettől függetlenül továbbra is előrejelezhetők. Legjobb tudásunk szerint ez az első olyan kutatás, amely ilyen szemszögből és ilyen mélységben vizsgálja a hallgatói teljesítmény változását a pandémia alatt.



\end{document}