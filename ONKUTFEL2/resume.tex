\documentclass[12pt]{article}
 
\usepackage[T1]{fontenc}     
\usepackage[utf8]{inputenc}  
\usepackage[magyar]{babel}   
\usepackage{lmodern}
\usepackage{caption}       
\usepackage{makecell}
\usepackage{boldline} 
 
\usepackage{amsmath,amssymb}
\usepackage{tikz, pgfplots}
\usetikzlibrary{positioning}
\usepackage{arydshln}
\usepackage{booktabs}
\usepackage{multirow}
\usepackage{booktabs} 
\usepackage{amsthm}          
\usepackage{graphicx}  
\usepackage{hyperref}  
\usepackage[export]{adjustbox}
\usepackage{wrapfig}
\usepackage{float}
\usepackage[a4paper, left=25mm, right=25mm, top=25mm, bottom=25mm]{geometry}

\usepackage{mathtools}

\usetikzlibrary{positioning,arrows,fit}

\renewcommand\vec[1]{\boldsymbol{#1}}


\DeclareMathOperator*{\argmax}{arg\,max}
\DeclareMathOperator*{\argmin}{arg\,min}   

\newtheorem{definition}{Definíció}[section]

\newcommand{\ra}[1]{\renewcommand{\arraystretch}{#1}}


\title{Covid-19-cel kapcsolatos tévinformációk azonosítása gépi tanulással}
\author{Köller Donát, BME TTK BSc. Szakdolgozat}
\date{\today}           
\captionsetup[table]{position=bottom}            
 
\linespread{1.2}

\begin{document}

%ide kéne a fancy fejléc

\center{\textbf{Elsőéves hallgatók pandémia előtti és alatti bemeneti
adatainak elemzése modern adattudományi
eszközökkel}}

Az oktatás fejlesztése és korszerűsítése minden időkben magas prioritású feladat volt. A néhány éve kitört koronavírus járvány miatt az oktatás a világ nagy részében távolléti formára állt át. Ebben az környezetben a diákok és hallgatók teljesítménye változást mutatott. Ennek fényében tekintettük célszerűnek az elsőéves BME VBK hallgatók eredményeinek vizsgálatát.

Kutatásunk célja adott bemeneti mutatók és a hallgatók további sikeressége közötti viszonyok feltárása és egy modell felépítése az utóbbi prediktálására. Egy ilyen modell birtokában a hallgató későbbi teljesítménye a tanulmányai kezdetén lenne megjósolható, így az eredménynek megfelelően vagy felzárkóztató, vagy tehetséggondozó, alaptanterven felüli foglalkozást biztosíthatnánk neki.



\end{document}